\documentclass{article}
% Uncomment the following line to allow the usage of graphics (.png, .jpg)
%\usepackage[pdftex]{graphicx}
% Comment the following line to NOT allow the usage of umlauts

\pagestyle{empty}
\usepackage[T2A]{fontenc}
\usepackage[utf8]{inputenc}
\usepackage[russian]{babel}
\usepackage{cmap}
\usepackage{amsthm}
\usepackage{amsmath}
\usepackage{units}
\usepackage{fancyhdr}
\usepackage{forloop}
\usepackage{amssymb}

\theoremstyle{definition}

\newtheorem*{theorem}{Теорема}
\newtheorem*{definition}{Определение}

\renewcommand{\thesection}{\arabic{section}}

\renewcommand{\headrulewidth}{0.4pt}
\renewcommand{\footrulewidth}{0.4pt}

\fancyhead[RO]{\bfseries \rightmark}
\fancyhead[LE]{\bfseries \rightmark}
\fancyhead[R]{Взятие производной}
%\fancyhead[LO, RE]{\today}
\fancyfoot[R]{Группа 573. Плюснин Павел}
%For multipage documents only!
%\fancyfoot[L]{page: \thepage}
%Uncomment this for 1-page sheets
\fancyfoot[L]{Расчет погрешностей}
\fancyfoot[C]{}


\renewcommand{\baselinestretch}{1.0}
\renewcommand\normalsize{\sloppypar}

\setlength{\topmargin}{-0.5in}
\setlength{\textheight}{9.1in}
\setlength{\oddsidemargin}{-0.3in}
\setlength{\evensidemargin}{-0.3in}
\setlength{\textwidth}{7in}
\setlength{\parindent}{0ex}
\setlength{\parskip}{1ex}
\pagestyle{fancy}

\newcounter{problemset}
\newcounter{example}
\newcounter{totalpages}
%Here you should set the total number of pages
\setcounter{totalpages}{10}

\def \topic { Расчет погрешности полученного результата }

\def \diam {\mathrm{diam}\;}

\def \R {\mathbb R}
\def \E {\mathbb E}

\begin{document}

	 		\begin{center}

	 			\newcommand{\HRule}{\rule{\linewidth}{0.5mm}}
	 			\HRule \\[0.2cm]
	 			{ \Huge \bfseries \topic} %\\[0.2cm]
	 			\HRule

	 		\end{center}
	 		\Large

            Определим погрешность полученной величины\\
	 		Для этого необходимо вычислить производную функции\\
	 		Данная задача не является слишком сложной, но требует внимательности и аккуратности. Давайте рассмотрим процесс взятия производной.\\
	 		{\LARGE \bfseries \center Математические преобразования: \\}
{\LARGE \bfseries \center Исходная формула: \\}\begin{center}$\frac{(x + 2y) ^ 2}{ \ln (x +  \sqrt {y})} = \frac{(12 + 2 \cdot 5.6) ^ 2}{ \ln (12 +  \sqrt {5.6})} = 201.974$\end{center}{\LARGE \bfseries \center Вычислим производную функции по переменной x \\}Вбивая в WolframAlpha, получаем производную
\begin{center}$ \sqrt {y}$ \end{center}равную
\begin{center}$(\frac{1}{2 \cdot ( \sqrt {y})}) \cdot 0$ \end{center}Преобразуя производную функции
\begin{center}$x +  \sqrt {y}$ \end{center}получаем
\begin{center}$1 + (\frac{1}{2 \cdot ( \sqrt {y})}) \cdot 0$ \end{center}Заметим, что производная фунцкции
\begin{center}$ \ln (x +  \sqrt {y})$ \end{center}равна
\begin{center}$(\frac{1}{x +  \sqrt {y}}) \cdot (1 + (\frac{1}{2 \cdot ( \sqrt {y})}) \cdot 0)$ \end{center}Легко заметить производную
\begin{center}$2y$ \end{center}равную
\begin{center}$0y + 2 \cdot 0$ \end{center}Преобразуя производную функции
\begin{center}$x + 2y$ \end{center}получаем
\begin{center}$1 + 0y + 2 \cdot 0$ \end{center}Опуская несложные выкладки, получим производную нижеуказанной функции,
\begin{center}$ \ln (x + 2y)$ \end{center}равную
\begin{center}$(\frac{1}{x + 2y}) \cdot (1 + 0y + 2 \cdot 0)$ \end{center}Заметим, что производная фунцкции
\begin{center}$2 \cdot ( \ln (x + 2y))$ \end{center}равна
\begin{center}$0 \cdot ( \ln (x + 2y)) + 2 \cdot ((\frac{1}{x + 2y}) \cdot (1 + 0y + 2 \cdot 0))$ \end{center}Легко заметить производную
\begin{center}$(x + 2y) ^ 2$ \end{center}равную
\begin{center}$((x + 2y) ^ 2) \cdot (0 \cdot ( \ln (x + 2y)) + 2 \cdot ((\frac{1}{x + 2y}) \cdot (1 + 0y + 2 \cdot 0)))$ \end{center}Очевидно, что производная
\begin{center}$\frac{(x + 2y) ^ 2}{ \ln (x +  \sqrt {y})}$ \end{center}равна
\begin{center}$\frac{(((x + 2y) ^ 2) \cdot (0 \cdot ( \ln (x + 2y)) + 2 \cdot ((\frac{1}{x + 2y}) \cdot (1 + 0y + 2 \cdot 0)))) \cdot ( \ln (x +  \sqrt {y})) - ((x + 2y) ^ 2) \cdot ((\frac{1}{x +  \sqrt {y}}) \cdot (1 + (\frac{1}{2 \cdot ( \sqrt {y})}) \cdot 0))}{( \ln (x +  \sqrt {y}))^ 2}$ \end{center}{\LARGE \bfseries \center Упростим полученную производную. \\}{\LARGE \bfseries \center Итак, производная функции равна \\}\begin{center}$\frac{d(\frac{(x + 2y) ^ 2}{ \ln (x +  \sqrt {y})})}{dx} = \frac{(((x + 2y) ^ 2) \cdot (2 \cdot (\frac{1}{x + 2y}))) \cdot ( \ln (x +  \sqrt {y})) - ((x + 2y) ^ 2) \cdot (\frac{1}{x +  \sqrt {y}})}{( \ln (x +  \sqrt {y}))^ 2}$ \end{center}{\LARGE \bfseries \center Вычислим конкретное значение по полученной формуле \\}\begin{center}$\frac{(((12 + 2 \cdot 5.6) ^ 2) \cdot (2 \cdot (\frac{1}{12 + 2 \cdot 5.6}))) \cdot ( \ln (12 +  \sqrt {5.6})) - ((12 + 2 \cdot 5.6) ^ 2) \cdot (\frac{1}{12 +  \sqrt {5.6}})}{( \ln (12 +  \sqrt {5.6}))^ 2} = 12.136$ \end{center}{\LARGE \bfseries \center Вычислим производную функции по переменной y \\}Совершенно ясно, что производная
\begin{center}$ \sqrt {y}$ \end{center}равна
\begin{center}$(\frac{1}{2 \cdot ( \sqrt {y})}) \cdot 1$ \end{center}Несложно заметить, что производная
\begin{center}$x +  \sqrt {y}$ \end{center}равна
\begin{center}$0 + (\frac{1}{2 \cdot ( \sqrt {y})}) \cdot 1$ \end{center}Нетрудно видеть, что производная
\begin{center}$ \ln (x +  \sqrt {y})$ \end{center}равна
\begin{center}$(\frac{1}{x +  \sqrt {y}}) \cdot (0 + (\frac{1}{2 \cdot ( \sqrt {y})}) \cdot 1)$ \end{center}Ясно, что производная этой функции
\begin{center}$2y$ \end{center}равна
\begin{center}$0y + 2 \cdot 1$ \end{center}Опуская несложные выкладки, получим производную нижеуказанной функции,
\begin{center}$x + 2y$ \end{center}равную
\begin{center}$0 + 0y + 2 \cdot 1$ \end{center}Опуская несложные выкладки, получим производную нижеуказанной функции,
\begin{center}$ \ln (x + 2y)$ \end{center}равную
\begin{center}$(\frac{1}{x + 2y}) \cdot (0 + 0y + 2 \cdot 1)$ \end{center}Опуская несложные выкладки, получим производную нижеуказанной функции,
\begin{center}$2 \cdot ( \ln (x + 2y))$ \end{center}равную
\begin{center}$0 \cdot ( \ln (x + 2y)) + 2 \cdot ((\frac{1}{x + 2y}) \cdot (0 + 0y + 2 \cdot 1))$ \end{center}Легко заметить производную
\begin{center}$(x + 2y) ^ 2$ \end{center}равную
\begin{center}$((x + 2y) ^ 2) \cdot (0 \cdot ( \ln (x + 2y)) + 2 \cdot ((\frac{1}{x + 2y}) \cdot (0 + 0y + 2 \cdot 1)))$ \end{center}Вбивая в WolframAlpha, получаем производную
\begin{center}$\frac{(x + 2y) ^ 2}{ \ln (x +  \sqrt {y})}$ \end{center}равную
\begin{center}$\frac{(((x + 2y) ^ 2) \cdot (0 \cdot ( \ln (x + 2y)) + 2 \cdot ((\frac{1}{x + 2y}) \cdot (0 + 0y + 2 \cdot 1)))) \cdot ( \ln (x +  \sqrt {y})) - ((x + 2y) ^ 2) \cdot ((\frac{1}{x +  \sqrt {y}}) \cdot (0 + (\frac{1}{2 \cdot ( \sqrt {y})}) \cdot 1))}{( \ln (x +  \sqrt {y}))^ 2}$ \end{center}{\LARGE \bfseries \center Упростим полученную производную. \\}{\LARGE \bfseries \center Итак, производная функции равна \\}\begin{center}$\frac{d(\frac{(x + 2y) ^ 2}{ \ln (x +  \sqrt {y})})}{dy} = \frac{(((x + 2y) ^ 2) \cdot (2 \cdot ((\frac{1}{x + 2y}) \cdot 2))) \cdot ( \ln (x +  \sqrt {y})) - ((x + 2y) ^ 2) \cdot ((\frac{1}{x +  \sqrt {y}}) \cdot (\frac{1}{2 \cdot ( \sqrt {y})}))}{( \ln (x +  \sqrt {y}))^ 2}$ \end{center}{\LARGE \bfseries \center Вычислим конкретное значение по полученной формуле \\}\begin{center}$\frac{(((12 + 2 \cdot 5.6) ^ 2) \cdot (2 \cdot ((\frac{1}{12 + 2 \cdot 5.6}) \cdot 2))) \cdot ( \ln (12 +  \sqrt {5.6})) - ((12 + 2 \cdot 5.6) ^ 2) \cdot ((\frac{1}{12 +  \sqrt {5.6}}) \cdot (\frac{1}{2 \cdot ( \sqrt {5.6})}))}{( \ln (12 +  \sqrt {5.6}))^ 2} = 33.7085$ \end{center}{\LARGE \bfseries \center Вычислим погрешность значения исходной формулы \\}\begin{center}$\Delta(\frac{(x + 2y) ^ 2}{ \ln (x +  \sqrt {y})}) = \sqrt{(\frac{d(\frac{(x + 2y) ^ 2}{ \ln (x +  \sqrt {y})})}{dx} \cdot \Delta(x)) ^ {2} + (\frac{d(\frac{(x + 2y) ^ 2}{ \ln (x +  \sqrt {y})})}{dy} \cdot \Delta(y)) ^ {2}} = \sqrt{(12.136 \cdot 1)^2 + (33.7085 \cdot 0.2)^2} = 13.8829$ \end{center}{\LARGE \bfseries \center Таким образом получаем, что \\}\begin{center}$\frac{(x + 2y) ^ 2}{ \ln (x +  \sqrt {y})} = 201.974 \pm (13.8829)$\end{center}{\large \bfseries \center Т.е полученная величина известна нам с ошикой в не более чем 6.87358\%\\}{\LARGE \bfseries \center Список использованной литературы:\\}1. Иванов Г.Е. Лекции по математическому анализу. Часть 1. (МФТИ - 2004г.)\\
2. Фихтенгольц Г.М. Курс дифференциального и интегрального исчисления (Первое издание 1948 г.)\\
{\LARGE \bfseries \center Спасибо за внимание!\\}
\end{document}